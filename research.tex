\documentclass[a4paper,14pt,ukrainian]{extarticle}
\usepackage[utf8]{inputenc}
\usepackage[ukrainian]{babel}
\usepackage{amsthm}
\usepackage{amssymb}
\usepackage{subfig}
\usepackage{tikz}
\usepackage{tkz-arith}
\usepackage{tkz-graph}
\usepackage{tkz-berge}

\newtheorem{problem}{Проблема}
\newtheorem{theorem}{Теорема}
\newtheorem{lemma}{Лема}

\begin{document}
    Нехай маємо граф $G(V,E)$, який розкладено відносно операції об’єднання графів на прості циклічні графи $G_1, G_2, \dots G_m$.
    \begin{problem}
        Проблема полягає у встановленні необхідних та достатніх умов для графів розкладу, які б гарантували існування гамільтонового циклу у заданому графі.
    \end{problem}
    Наївна умова для двох простих циклічних графів - повинні існувати щонайменше 2 спільні вершини, які є необхідними як ``містки'', через один із яких забезпечувався б перехід з одного графу до іншого, а через другий - повернення до вихідної точки.
    \begin{theorem}
    \label{JoinOfTwoCycleGraphs}
        Для того, щоб об’єднання двох простих циклічних графів $G_1$ та $G_2$ було гамільтоновим графом необхідно та достатньо, щоб у них існувалo одне і тільки одне спільне ребро.
    \end{theorem}
    \begin{proof}
        Розглянемо усі можливі варіанти об’єднання двох простих циклічних графів $G_1(V=\lbrace a_0, a_1, \dots, a_k \rbrace ,E= \lbrace (a_0; a_1), \dots, (a_k; a_0) \rbrace)$, $G_2(V= \lbrace b_0, b_1, \dots, b_l \rbrace, E= \lbrace (b_0; b_1), \dots, (b_k; b_0) \rbrace )$:
        \begin{enumerate}
            \item Графи не мають спільних вершин. В такому випадку, очевидно, що їх об’єднання не може мати гамільтонового циклу.
                \begin{figure}[h]
                    \caption{Графи $G_1$ та $G_2$ не мають спільних вершин}
                    \centering
                    \subfloat[$G_1$]{
                    \begin{tikzpicture}
                        \SetVertexMath
                        \grEmptyCycle[RA=2,prefix=a]{8}%
                        \EdgeInGraphLoop{a}{8}
                    \end{tikzpicture}
                    }
                    \subfloat[$G_2$]{
                    \begin{tikzpicture}
                        \SetVertexMath
                        \grEmptyCycle[RA=2,prefix=b]{6}%
                        \EdgeInGraphLoop{b}{6}
                    \end{tikzpicture}
                    }
                \end{figure}
            \item Графи мають лише одну спільну вершину $v_c$. В такому випадку можливі два випадки можливі такі ситуації:
                \begin{enumerate}
                    \item Перевірка на наявність ГЦ розпочинається у спільній вершині $v_c$. Тоді наступними вершинами мають бути вершини графу $G_1$ чи $G_2$, які врешті замкнуться (через циклічну структуру графа) у вершині $v_c$, а вершини іншого графу будуть проігнорованими і таким чином ГЦ не існує.
                    \item Перевірка починається із вершини $v_0 \in V_1 \setminus v_c$ або $v_0 \in V_2 \setminus v_c$. Не порушуючи загальність нехай $v_0 \in V_1 = a_i$ для деякого $i$.
                        Проходячи одна за одною вершини графа $G_1$, щонайбільше через $k$ кроків потрапимо у вершину $v_c$. Після цього можливі 2 варіанти:
                        \begin{enumerate}
                            \item Піти далі по графу $G_1$ замкнувши цикл у вершині $v_0$. Таким чином $v \in V_2 \setminus v_c$ не пройдені, а отже ГЦ не має місце.
                            \item Перейти на граф $G_2$ замкнувши цикл у вершині $v_c$. Тоді залишаться не пройденими вершини графа $G_1$ а саме: $a_{i+2}, \dots, v_0$.
                        \end{enumerate}
                \end{enumerate}
                \begin{figure}[h]
                    \caption{Графи із однією спільною вершиною}
                    \centering
                    \begin{tikzpicture}
                        \SetVertexMath
                        \begin{scope}[node distance=2cm,rotate=135]
                            \Vertices*{circle}{a_k,a_0,a_i,a_{i+2}}
                            \Edge(a_k)(a_0)
                            \Edge[style=dotted](a_k)(a_{i+2})
                            \Edge[style=dotted](a_0)(a_i)
                            \SOEA(a_{i+2}){v_c}
                            \Edge(a_{i+2})(v_c)
                            \Edge(a_i)(v_c)
                            \NOEA(v_c){b_{j+2}}
                            \SOEA(v_c){b_j}
                            \EA(b_{j+2}){b_l}
                            \EA(b_j){b_0}
                            \Edge(v_c)(b_j)
                            \Edge(v_c)(b_{j+2})
                            \Edge(b_0)(b_l)
                            \Edge[style=dotted](b_{j+2})(b_l)
                            \Edge[style=dotted](b_j)(b_0)
                        \end{scope}
                    \end{tikzpicture}
                \end{figure}
            \item Графи мають 2 спільні суміжні вершини (спільне ребро).
                \begin{figure}[h]
                    \caption{Графи із спільним ребром}
                    \centering
                    \begin{tikzpicture}
                        \SetVertexMath
                        \begin{scope}[node distance=2cm,rotate=-135]
                            \Vertices*{circle}{a_0,v_{c1},v_{c2},a_k}
                            \Edge(a_0)(a_k)
                            \Edge[style=dotted](a_0)(v_{c1})
                            \Edge[style=dotted](a_k)(v_{c2})
                            \Edge(v_{c1})(v_{c2})
                            \EA(v_{c2}){b_l}
                            \EA(v_{c1}){b_0}
                            \Edge[style=dotted](v_{c2})(b_l)
                            \Edge[style=dotted](v_{c1})(b_0)
                            \Edge(b_l)(b_0)
                        \end{scope}
                    \end{tikzpicture}
                \end{figure}
                Яка б вершина $v_0$ не була початковою, через кільцеву структуру графа, через max(k,l) вершин зустрінеться $v_{c1}$ чи $v_{c2}$.
                Тоді можна перейти на наступний граф, пройти його до наступної спільної вершини $v_{c2}$ чи $v_{c1}$ відповідно і замкнути цикл у початковій вершині $v_0$.
            \item Графи мають 2 або більше несуміжні вершини або 3 і більше суміжних вершини.
                \begin{figure}[h]
                    \label{jg2+na3+a}
                    \caption{Графи мають 2+ несуміжні або 3+ суміжні вершини}
                    \centering
                    \begin{tikzpicture}
                        \SetVertexMath
                        \begin{scope}[node distance=2cm,rotate=180]
                            \Vertex{a_0}
                            \NOEA(a_0){v_{c0}}
                            \SOEA(a_0){v_{cm}}
                            \SOEA(v_{c0}){b_0}
                            \begin{scope}[style=dotted]
                                \Edges(a_0,v_{c0},b_0,v_{cm},a_0)
                                \Edge(v_{c0})(v_{cm})
                            \end{scope}
                        \end{scope}
                    \end{tikzpicture}
                \end{figure}
                Для зручності об’єднання неоднакових графів, що мають 2 або більше спільні несуміжні вершини чи 3 або більше суміжні вершини можна зобразити, як ізоморфний їм граф зображений на Рис. \ref{jg2+na3+a}.
                Існування гамільтонового циклу вимагає існування циклу між будь-якими двома вершинами графу.
                Розглянемо вершини $a_0$ та $b_0$.
                Зважаючи на циклічну природу графів $G_1$ та $G_2$ існує по 2 простих шляхи з $a_0$ до $v_{c0}$, з $a_0$ до $v_{cm}$, з $v_{c0}$ до $b_0$ та $v_{cm}$ до $b_0$.
                Використавши закон множення теорії комбінаторики, отримуємо існуснування $2*2=4$ простих шляхів з $a_0$ до $b_0$.
                Розглянемо ці шляхи:
                \begin{enumerate}
                    \item{$a_0$ -> $v_{c0}$ -> $b_0$:} Потрапивши у $b_0$ повернутися до $a_0$, не використовуючи $v_{c0}$ можна лише через $v_{cm}$.
                        Але, в такому випадку, можна, або пройти вершини між $v_{cm}$ та $v_{c0}$, таким чином не маючи можливості рухатись далі (так як $v_{c0}$ вже була використана), або пройти до вершини $a_0$, залишивши непройденими вершини між $v_{c0}$ та $v_{cm}$.
                    \item{$a_0$ -> $v_{cm}$ -> $b_0$:} Ситуація дзеркальна до попереднього випадку.
                    \item{$a_0$ -> $v_{c0}$ -> $v_{cm}$ -> $b_0$:} В такому разі повернення до $a_0$ взагалі неможливе, так як і $v_{c0}$, і $v_{cm}$ вже використані.
                    \item{$a_0$ -> $v_{cm}$ -> $v_{c0}$ -> $b_0$:} Ситуація дзеркальна до попереднього пояснення.
                \end{enumerate}
        \end{enumerate}
        Таким чином, для того, щоб об’єднання двох простих циклічних графів $G_1$ та $G_2$ було гамільтоновим графом, необхідно та достатньо, щоб у них існували 2 спільні суміжні вершини.
    \end{proof}
    Також, не важко помітити, що гамільтоновий цикл може існувати лише у графі, який можна розкласти відносно операції об’єднання на прості циклічні підграфи.
    Адже, якщо це не так, то між деякими 2 вершинами не буде існувати шляху взагалі (у випадку, якщо існує більше однієї компоненти зв’язності), або ж він буде лише одностороннім (якщо ``залишок'' після розкладу графу відносно операції з’єднання буде шляхом), що, очевидно, не задовольняє умову існування циклу взагалі, i відповідно гамільтонового також.
    \begin{lemma}
    \label{FullDecompositionOnly}
        Гамільтоновий цикл може існувати лише у графі $G$, який можна розкласти відносно операції об’єднання на прості циклічні підграфи.
    \end{lemma}
    \begin{proof}
        Нехай маємо граф $G$.
        Розкладемо його на максимально допустиму кількість простих циклічних графів.
        Нехай це будуть графи $G_1, \cdots G_k$.
        Віднімемо їх відносно операції об’єднання від $G$.
        Можливі декілька можливих результатів:
        \begin{itemize}
            \item незв’язний граф.
                У такому випадку існування гамільтонового циклу очевидно не можливе
            \item зв’язний граф. 
                Даний граф не може мати цикла (адже тоді він міг би бути розкладений на прості циклічні підграфи), а, отже, не може бути частиною гамільтонового циклу (так як існує лише один шлях до принаймні однієї із вершин даного підграфа, а ,отже, дана вершина не може належати гамільтоновому циклу).
        \end{itemize}
    \end{proof}
    \begin{lemma}
        Об’єднання гамільтонового графу із простим циклічним графом, що мають лише одне спільне ребро також є гамільтоновим графом.
    \end{lemma}
    \label{JoinOfHamiltonianAndCycleGraphs}
    \begin{proof}
        Доведення аналогічне доведенню теореми \ref{JoinOfTwoCycleGraphs}.
    \end{proof}
    \begin{lemma}
        Якщо у графі $G$ існує простий циклічний граф $G_i$, що не має спільного ребра із жодним іншим простим циклічним графом із розкладу графа $G$ на прості циклічні підграфи відносно операції об’єднання графів, то граф $G$ - не гамільтоновий.
    \end{lemma}
    \begin{proof}
        Розділимо граф $G$ відносно операції об’єднання на дві частини $G_i$ та $G_r$.
        \begin{enumerate}
            \item Якщо $G_r$ - простий циклічний граф, то згідно теореми \ref{JoinOfTwoCycleGraphs} $G$ - не гамільтоновий.
            \item Якщо $G_r$ - не простий циклічний граф, то згідно леми \ref{FullDecompositionOnly} $G$ - не гамільтоновий.
        \end{enumerate}
    \end{proof}
    \begin{theorem}
        Об’єдання простих циклічних графів $G_1, \cdots , G_k$ є гамільтоновим графом тоді і тільки тоді, якщо їхнє представлення у вигляді графу, вершинами якого є дані графи, а ребра з’єднують лише вершини $v_i$ та $v_j$, які представляють графи $G_i$ та $G_j$ відповідно, що мають між собою лише одне спільне ребро, має гамільтоновий шлях.
    \end{theorem}
    \begin{proof}
        Доведення достатності за індукцією:
        \begin{itemize}
            \item База індукції - простий циклічний граф є гамільтоновим графом
            \item Нехай маємо об’єднання $i$ простих циклічних графів, що утворюють гамільтоновий граф
            \item Тоді об’єднання із $i+1$-им простим циклічним графом також буде гамільтоновим графом згідно із лемою \ref{JoinOfHamiltonianAndCycleGraphs}
        \end{itemize}
        Об’єднані таким чином усі прості підграфи згідно наведеного вище індуктивного доведення утворюють гамільтоновий граф.
        Оскільки у графі їх представлення існує гамільтоновий шлях то таке об’єднання обов’язково існує.
        
        Необхідність.
        Від супротивного.
        Нехай не існує гамільтонового шляху між вершинами графу представлення $G$.
        Тоді 
        \begin{enumerate}
        \item Граф $G$ має більш ніж одну компоненту зв’язності, а, отже, не може бути гамільтоновим графом.
        \item Граф $G$ має лише шляхи, які потребують проходження принаймні однієї вершини $v_i$ більше одного разу, що в свою чергу означає необхідність повторного проходження вершин графа $G_i$ розбиття графа $G$ на прості циклічні підграфи відносно операції об’єднання. А це суперечить умовам гамільтоновості графа $G$. Отже, $G$ - не гамільтоновий.
        \end{enumerate}
    \end{proof}
    \begin{theorem}
        Для того, щоб граф був гамільтоновим необхідно та достатньо, щоб для нього існував розклад відносно операції з’єднання на циклічні підграфи, такий, що представлений у вигляді графу, вершинами якого є циклічні підграфи із ребрами між вершинами, що представляють підграфи, у яких існує лише одне спільне ребро, має гамільтоновий шлях.
    \end{theorem}
    \begin{proof}
    \end{proof}

    Таким чином алгоритм перевірки наявності гамільтонового циклу можна описати такими кроками:
    \begin{enumerate}
        \item Прономерувати вершини графа застосувавши алгоритм BFS, де $v_0 = \min{deg(v)}$.
        \item Розкласти граф на мінімальні циклічні підграфи.
            Якщо це неможливо - гамільтонового циклу не існує.
        \item Починаючи із вершин $v_0$ пройти усі рівні, на кожному із яких фіксувати внутрішні ребра та ``горизонтальні ребра'' між рівнями.
        \item Побудувати новий граф $G_1$, вершинами якого є всі прості циклічні підграфи розкладу графа $G$, а ребра існують лише між тими із них, які мають лише 2 спільні суміжні вершини.
            Якщо отриманий граф не є зв’язним - $G$ - не гамільтоновий.
        \item Для того, щоб граф $G$ був гамільтоновим, необхідно, щоб граф $G_1$ мав гамільтоновий шлях.

    \end{enumerate}
\end{document} 
